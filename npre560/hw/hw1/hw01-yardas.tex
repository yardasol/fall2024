% use the answers clause to get answers to print; otherwise leave it out.
\documentclass[11pt,addpoints,answers]{exam}
%\documentclass[11pt,addpoints]{exam}
\RequirePackage{amssymb, amsfonts, amsmath, latexsym, verbatim, xspace, 
setspace, wasysym}
\usepackage{graphicx}

% By default LaTeX uses large margins.  This doesn't work well on exams; problems
% end up in the "middle" of the page, reducing the amount of space for students
% to work on them.
\usepackage[margin=1in]{geometry}
\usepackage{enumerate}
\usepackage[hidelinks]{hyperref}

% Here's where you edit the Class, Exam, Date, etc.
\newcommand{\class}{NPRE 560}
\newcommand{\term}{Fall 2024}
\newcommand{\assignment}{HW 1}
\newcommand{\duedate}{2024.09.09}
\newcommand{\StudentName}{Oleksandr Yardas} %Please include your name here

\newcommand{\nth}{n\ensuremath{^{\text{th}}} }
\newcommand{\ve}[1]{\ensuremath{\mathbf{#1}}}
\newcommand{\Macro}{\ensuremath{\Sigma}}
\newcommand{\vOmega}{\ensuremath{\hat{\Omega}}}

% For an exam, single spacing is most appropriate
\singlespacing
% \onehalfspacing
% \doublespacing

% For an exam, we generally want to turn off paragraph indentation
\parindent 0ex

%\unframedsolutions

\begin{document} 

% These commands set up the running header on the top of the exam pages
\pagestyle{head}
\firstpageheader{}{}{}
\runningheader{\class}{\assignment\ - Page \thepage\ of \numpages}{Due \duedate}
\runningheadrule

\class \hfill \StudentName \hfill \term \\
\assignment \hfill Due \duedate\\
\rule[1ex]{\textwidth}{.1pt}
%\hrulefill

%%%%%%%%%%%%%%%%%%%%%%%%%%%%%%%%%%%%%%%%%%%%%%%%%%%%%%%%%%%%%%%%%%%%%%%%%%%%%%%%%%%%%
%%%%%%%%%%%%%%%%%%%%%%%%%%%%%%%%%%%%%%%%%%%%%%%%%%%%%%%%%%%%%%%%%%%%%%%%%%%%%%%%%%%%%
\begin{itemize}
        \item Show your work. 
        \item This work must be submitted online as a \texttt{.pdf} through Compass2g.
        \item Work completed with LaTeX or Jupyter earns 1 extra point. Submit 
                source file (e.g. \texttt{.tex} or \texttt{.ipynb}) along with 
                the \texttt{.pdf} file.
        \item If this work is completed with the aid of a numerical program 
                (such as Python, Wolfram Alpha, or MATLAB) all scripts and data 
                must be submitted in addition to the \texttt{.pdf}.
        \item If you work with anyone else, document what you worked on together.
\end{itemize}
\rule[1ex]{\textwidth}{.1pt}

% ---------------------------------------------
\begin{questions}
        \question[20] Review any resource on the use of \texttt{git} and 
                GitHub until you feel confident that you will be able to use 
                GitHub Classroom to turn in your three computational projects. 
                Create a GitHub account and provide your GitHub 
                username as an answer to this question.
                \begin{solution}
                    My GithHub username is \verb.yardasol.
                \end{solution}

        % ---------------------------------------------
        \question[10] (Ott Review Question 1.1)
        State three areas of kinetics or dynamics applications.
                \begin{solution}
                    Kinetics refers to time-dependent phenomena over a short
                    time scale, on the order of milliseconds to seconds, without
                    considreing feedback effects. Dynamics is the same but with
                    consideration of feedback effects. Three areas where
                    we apply kinetic theory are:
                    \begin{enumerate}
                        \item Reactor stabiltiy analysis w.r.t. neutron flux
                            changes (kinetics and dynamics)
                        \item Accident transients (e.g. LOCA) (dynamics)
                        \item Time-dependent experiments, such as the Dragon
                            criticality experiment or a neutron pulse.
                    \end{enumerate}efers to short time phenomena with feedback.
                \end{solution}

        % ---------------------------------------------
        \question[10] (Ott Review Question 1.4)
        What is the main difference in the balance equations for the neutron 
        flux in reactor dynamics versues fuel cycle analysis?
                \begin{solution}
                    In the balance equation for fuel cycle analysis, we must
                    consider nuclide transmuation, where in the balance equation
                    for reactor dynamics we assume the nuclide composition to
                    be static.
                \end{solution}

        % ---------------------------------------------
        \question (Ott Homework Question 2.1) 
        \begin{parts} 
                \part[10] Calculate the average 
                energy $\bar{E_k}$, of the delayed neutron groups 1 through 4, 
                using the emission spectra $\chi_{dk}(E)$ given in table 2-V.
                \begin{solution}
                    I used Python to calculate the average energies:
                    \begin{itemize}
                        \item $\overline{E}_{1} = 10.34$ keV
                        \item $\overline{E}_{2} = 18.62$ keV
                        \item $\overline{E}_{3} = 15.10$ keV
                        \item $\overline{E}_{4} = 17.58$ keV
                    \end{itemize}
                \end{solution}
                \part[10] 
                Compare these values with $\bar{E}$ for the total 
                $\chi(E)$ given in the same table and with $\bar{E_k}$ of Table 
                2-IV.
                \begin{solution}
                    I used Python to calculate $\overline{E} = 164.14$ keV. This
                    is much greater than the average delayed neutron energies
                    for all delayed groups, which is what we expect (expand on
                    this from ch 2.1). The average energies for neutrons from
                    delayed groups 1 through 4 calculated from
                    Table 2-V are an order of magnitude smaller than the average
                    energies for the same delayed groups given in Table 2-IV.
                    (why? did I do calculation wrong?)
                \end{solution}
        \end{parts}

        % ---------------------------------------------
        \question[10] (Ott Review Question 2.5)
        Give approximately the total precursor yields of $^{235}U$, $^{238}U$, 
        and $^{239}Pu$.
                \begin{solution}
                        solution here
                \end{solution}

        % ---------------------------------------------
        \question[10] (Ott Review Question 2.8)
        How many delayed neutron groups (families) are generally used per 
        fissioning isotope?
                \begin{solution}
                        solution here
                \end{solution}

        % ---------------------------------------------
        \question[10] (Ott Review Question 2.9)
        What is the disadvantage of using isotope-dependent decay constants?
                \begin{solution}
                        solution here
                \end{solution}

        % ---------------------------------------------
        \question[10] (Ott Review Question 2.12)
        Give the approximate mean lifetime of the slowest and the fastest 
        decaying precursor group.
                \begin{solution}
                        solution here
                \end{solution}

\end{questions}



%\bibliographystyle{plain}
%\bibliography{hw01}
\end{document}
