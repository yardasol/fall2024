% use the answers clause to get answers to print; otherwise leave it out.
\documentclass[11pt,addpoints,answers]{exam}
%\documentclass[11pt,addpoints]{exam}
\RequirePackage{amssymb, amsfonts, amsmath, latexsym, verbatim, xspace, 
setspace, wasysym}
\usepackage{graphicx}

% By default LaTeX uses large margins.  This doesn't work well on exams; problems
% end up in the "middle" of the page, reducing the amount of space for students
% to work on them.
\usepackage[margin=1in]{geometry}
\usepackage{enumerate}
\usepackage[hidelinks]{hyperref}

% Here's where you edit the Class, Exam, Date, etc.
\newcommand{\class}{NPRE 560}
\newcommand{\term}{Fall 2024}
\newcommand{\assignment}{HW 2}
\newcommand{\duedate}{2024.09.30}
%\newcommand{\timelimit}{50 Minutes}
\newcommand{\StudentName}{Firstname Lastname} %Please include your name here

\newcommand{\nth}{n\ensuremath{^{\text{th}}} }
\newcommand{\ve}[1]{\ensuremath{\mathbf{#1}}}
\newcommand{\Macro}{\ensuremath{\Sigma}}
\newcommand{\vOmega}{\ensuremath{\hat{\Omega}}}

% For an exam, single spacing is most appropriate
\singlespacing
% \onehalfspacing
% \doublespacing

% For an exam, we generally want to turn off paragraph indentation
\parindent 0ex

%\unframedsolutions

\begin{document} 

% These commands set up the running header on the top of the exam pages
\pagestyle{head}
\firstpageheader{}{}{}
\runningheader{\class}{\assignment\ - Page \thepage\ of \numpages}{Due \duedate}
\runningheadrule

\class \hfill \StudentName \hfill \term \\
\assignment \hfill Due \duedate\\
\rule[1ex]{\textwidth}{.1pt}
%\hrulefill

%%%%%%%%%%%%%%%%%%%%%%%%%%%%%%%%%%%%%%%%%%%%%%%%%%%%%%%%%%%%%%%%%%%%%%%%%%%%%%%%%%%%%
%%%%%%%%%%%%%%%%%%%%%%%%%%%%%%%%%%%%%%%%%%%%%%%%%%%%%%%%%%%%%%%%%%%%%%%%%%%%%%%%%%%%%
\begin{itemize}
        \item Show your work. 
        \item This work must be submitted online as a \texttt{.pdf} through 
                Canvas.
        \item Work completed with LaTeX or Jupyter earns 1 extra point. Submit 
                source file (e.g. \texttt{.tex} or \texttt{.ipynb}) along with 
                the \texttt{.pdf} file.
        \item If this work is completed with the aid of a numerical program 
                (such as Python, Wolfram Alpha, or MATLAB) all scripts and data 
                must be submitted in addition to the \texttt{.pdf}.
        \item If you work with anyone else, document what you worked on together.
\end{itemize}
\rule[1ex]{\textwidth}{.1pt}

% ---------------------------------------------
\begin{questions}
        \question (Ott Problem 3.3)
        \begin{parts}
                \part[5] Define a one-group $\nu\Sigma_f$ based on the two 
                group values 
                $\nu \Sigma_{f1} = 0.015 \left[\frac{1}{cm}\right]$ and 
                $\nu \Sigma_{f2} = 0.3 \left[\frac{1}{cm}\right]$.
                \begin{solution}
                        solution here
                \end{solution}

                \part[5] Calculate $\Lambda$ with $\bar{v}$ and with 
                $\bar{\frac{1}{v}}$.
                \begin{solution}
                        solution here
                \end{solution}

                \part[5] Discuss the results.
                \begin{solution}
                        solution here
                \end{solution}

        \end{parts}

        % ---------------------------------------------
        \question (Ott Review 3.8)
        \begin{parts}
                \part[5] Give the formula for $\beta_{eff}$ in the one 
        group approximation.
                \begin{solution}
                        solution here
                \end{solution}

                \part[10] Which physical fact is described by $\beta_{eff}$ in 
                this approximation?
                \begin{solution}
                        solution here
                \end{solution}
        \end{parts}


        % ---------------------------------------------
        \question[10] (Ott Review 3.10) Why is the diffusion equation a reasonable 
        approximation for kinetics in large reactors?
        \begin{solution}
                solution here.
        \end{solution}

        % ---------------------------------------------
        \question (Ott 4.4) Consider a perturbation of $+\delta \Sigma_a$ 
        for $r< r_a$ in a critical sphere. Assume $r_a$ is much less than the 
        critical radius, R. Find the corresponding change in the reactivity 
        using the unperturbed flux
        \begin{parts}
                \part[5] from reaction rates
                        \begin{solution}
                                solution here.
                        \end{solution}
                \part[5] from the first-order perturbation formula for the 
                one-group approximation.
                        \begin{solution}
                                solution here.
                        \end{solution}
        \end{parts}

        % ---------------------------------------------
        \question[5] (Ott Review 5.4) What does ``exact'' mean in the context 
        of exact point kinetics equations?
        \begin{solution}
                solution here.
        \end{solution}
        

        % ---------------------------------------------
        \question (Ott Problem 6.1) Inhour equation.
        \begin{parts}
                \part[10] Find the stable and prompt-period branches for 
                $^{235}U$  fuel with
                $\Lambda = 10^{-4}$s,  
                $\Lambda = 10^{-5}$s, and  
                $\Lambda = 4\times10^{-7}$s. Data is given in Table 2-III.
                \begin{solution}
                        solution here.
                \end{solution}

                \part[10] Find $\rho(\alpha)$ in the one-delay-group 
                approximation with $\lambda = \bar{\lambda}$.
                \begin{solution}
                        solution here.
                \end{solution}

        \end{parts}

        % ---------------------------------------------
        \question[5] (Ott Review 6.13) Which conditions have to be fulfilled to 
        yield an asymptotic transient with a ``stable'' period?
        \begin{solution}
                solution here.
        \end{solution}

        
        % ---------------------------------------------
        \question[10] (Ott Review 6.18b) Estimate the stable period for an 
        asymptotic transient following a reactivity insertion of $\rho = 
        1.25\cent$.
        \begin{solution}
                solution here.
        \end{solution}


        % ---------------------------------------------
        \question[10] (Ott Review 6.19) Give a list of six to seven 
        kinetics equations with decreasing sophistication of the description of 
        delayed neutrons.
        \begin{solution}
                solution here.
        \end{solution}
        
\end{questions}



%\bibliographystyle{plain}
%\bibliography{hw01}
\end{document}
