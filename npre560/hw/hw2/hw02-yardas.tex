% use the answers clause to get answers to print; otherwise leave it out.
\documentclass[11pt,addpoints,answers]{exam}
%\documentclass[11pt,addpoints]{exam}
\RequirePackage{amssymb, amsfonts, amsmath, latexsym, verbatim, xspace, 
setspace, wasysym}
\usepackage{graphicx}

% By default LaTeX uses large margins.  This doesn't work well on exams; problems
% end up in the "middle" of the page, reducing the amount of space for students
% to work on them.
\usepackage[margin=1in]{geometry}
\usepackage{enumerate}
\usepackage[hidelinks]{hyperref}

% Here's where you edit the Class, Exam, Date, etc.
\newcommand{\class}{NPRE 560}
\newcommand{\term}{Fall 2024}
\newcommand{\assignment}{HW 2}
\newcommand{\duedate}{2024.09.30}
%\newcommand{\timelimit}{50 Minutes}
\newcommand{\StudentName}{Oleksandr Yardas} %Please include your name here

\newcommand{\nth}{n\ensuremath{^{\text{th}}} }
\newcommand{\ve}[1]{\ensuremath{\mathbf{#1}}}
\newcommand{\Macro}{\ensuremath{\Sigma}}
\newcommand{\vOmega}{\ensuremath{\hat{\Omega}}}

% For an exam, single spacing is most appropriate
\singlespacing
% \onehalfspacing
% \doublespacing

% For an exam, we generally want to turn off paragraph indentation
\parindent 0ex

%\unframedsolutions

\begin{document} 

% These commands set up the running header on the top of the exam pages
\pagestyle{head}
\firstpageheader{}{}{}
\runningheader{\class}{\assignment\ - Page \thepage\ of \numpages}{Due \duedate}
\runningheadrule

\class \hfill \StudentName \hfill \term \\
\assignment \hfill Due \duedate\\
\rule[1ex]{\textwidth}{.1pt}
%\hrulefill

%%%%%%%%%%%%%%%%%%%%%%%%%%%%%%%%%%%%%%%%%%%%%%%%%%%%%%%%%%%%%%%%%%%%%%%%%%%%%%%%%%%%%
%%%%%%%%%%%%%%%%%%%%%%%%%%%%%%%%%%%%%%%%%%%%%%%%%%%%%%%%%%%%%%%%%%%%%%%%%%%%%%%%%%%%%
\begin{itemize}
        \item Show your work. 
        \item This work must be submitted online as a \texttt{.pdf} through 
                Canvas.
        \item Work completed with LaTeX or Jupyter earns 1 extra point. Submit 
                source file (e.g. \texttt{.tex} or \texttt{.ipynb}) along with 
                the \texttt{.pdf} file.
        \item If this work is completed with the aid of a numerical program 
                (such as Python, Wolfram Alpha, or MATLAB) all scripts and data 
                must be submitted in addition to the \texttt{.pdf}.
        \item If you work with anyone else, document what you worked on together.
\end{itemize}
\rule[1ex]{\textwidth}{.1pt}

% ---------------------------------------------
\begin{questions}
        \question (Ott Problem 3.3)
        \begin{parts}
                \part[5] Define a one-group $\nu\Sigma_f$ based on the two 
                group values 
                $\nu \Sigma_{f1} = 0.015 \left[\frac{1}{cm}\right]$ and 
                $\nu \Sigma_{f2} = 0.3 \left[\frac{1}{cm}\right]$.
                \begin{solution}
                    Let's assume the weigting of each energy group is $0.5$. We
                    can then calculate 
                    \begin{align*}
                        \nu \Sigma_{f} &= 0.5 \nu \Sigma_{f1} + 0.5 \nu
                            \Sigma_{f2} \\
                        &= 0.5 * 0.015 \text{cm}^{-1} + 0.5 * 0.3
                            \text{cm}^{-1}\\
                        &= 0.1575 \text{cm}^{-1}
                    \end{align*}
                \end{solution}

                \part[5] Calculate $\Lambda$ with $\bar{v}$ and with 
                $\bar{\frac{1}{v}}$.
                \begin{solution}
                    Suppose our one-group neutrons are at 0.025 eV. Then their
                    velocity is
                    \begin{align*}
                        \bar{v} &= \sqrt{\frac{2E}{m_{n}}} \\
                        &= \sqrt{\frac{2 * 0.025 \text{eV} * 1.602 * 10^{-19}
                            \frac{\text{J}}{\text{eV}}}{1.6749 * 10^{-27}
                            \text{kg}}} \\
                        &= 2186 \text{m/s}
                    \end{align*}
                \end{solution}

                \part[5] Discuss the results.
                \begin{solution}
                        solution here
                \end{solution}

        \end{parts}

        % ---------------------------------------------
        \question (Ott Review 3.8)
        \begin{parts}
                \part[5] Give the formula for $\beta_{eff}$ in the one 
        group approximation.
                \begin{solution}
                    In one group, we have
                    \begin{equation}
                        \beta_\text{eff} = \frac{\nu_{d} \Sigma_{f}}{\nu
                        \Sigma_{f}}
                    \end{equation}
                    where $\nu_{d}$ is the delayed neutron yield from fission.
                \end{solution}

                \part[10] Which physical fact is described by $\beta_{eff}$ in 
                this approximation?
                \begin{solution}
                    $\beta_\text{eff}$ describes the fraction of delayed
                    neutrons produced from fission.
                \end{solution}
        \end{parts}


        % ---------------------------------------------
        \question[10] (Ott Review 3.10) Why is the diffusion equation a reasonable 
        approximation for kinetics in large reactors?
        \begin{solution}
            According to Ott, multigroup diffusion theory using the appropriate
            cell averaged group constants is a good approximation for reactor
            statics, but has errors near reflectors and blankets at the edge of
            the core (as well as at control rods). For kinetics problems in
            large reactors, the reflector and blanket regions are of low
            importance for reactivity, as the adjoint flux in these regions will
            be small in a large reactor. Since the reactivity governs the
            time-dependence of the flux, applying the diffusion approxmation for
            deriving kinetics equations incurrs acceptably low errors.
        \end{solution}

        % ---------------------------------------------
        \question (Ott 4.4) Consider a perturbation of $+\delta \Sigma_a$ 
        for $r< r_a$ in a critical sphere. Assume $r_a$ is much less than the 
        critical radius, R. Find the corresponding change in the reactivity 
        using the unperturbed flux
        \begin{parts}
                \part[5] from reaction rates
                        \begin{solution}
                            We can define reactivity $\rho$ based on the
                            multiplication factor $k$ as
                            \begin{equation}
                                \rho = 1 - \frac{1}{k}
                            \end{equation}
                            Let the inital state of the sphere be inidcated by
                            superscript $0$, and the perturbed state by
                            superscript $1$. Assuming we are starting at
                            criticality, we have $k^{0} = 1$ and $\rho^{0}=0$.
                            We can also define $k^{0}$ based on reaction rates
                            as 
                            \begin{equation}
                                k^{0} = \frac{\text{neutrons
                                production}}{\text{neutron loss}} = \frac{\nu
                                \Sigma_{f}}{\Sigma_{a} +\Sigma_{s}} 
                            \end{equation}
                            It then follows that 
                            \begin{equation}
                                \Delta \rho = \rho^{1} - \rho^{0} = \rho^{1} = 1 -\frac{1}{k^1}
                            \end{equation}
                            We can in turn define $k^{1}$ based on reaction rates as
                            \begin{equation}
                                k^{1} = \frac{\text{neutrons
                                production}}{\text{neutron loss}} = \frac{\nu
                                \Sigma_{f}}{\Sigma_{a} + \delta \Sigma_{a} + \Sigma_{s}} 
                            \end{equation}
                            Plugging this into the equation for $\delta \rho$,
                            we have
                            \begin{align*}
                                \delta \rho &= 1 - \frac{\Sigma_{a} + \delta
                                    \Sigma_{a} + \Sigma_{s}}{\nu \Sigma_{f}}\\
                                &= 1 - \frac{\Sigma_{a} +
                                    \Sigma_{s}}{\nu\Sigma_{f}} - \frac{\delta
                                    \Sigma_{a}}{\nu \Sigma_{f}} \\
                                &= 1 - \frac{1}{k^{0} } - \frac{\delta
                                    \Sigma_{a}}{\nu \Sigma_{f}} \\
                                &=- \frac{\delta
                                    \Sigma_{a}}{\nu \Sigma_{f}}
                            \end{align*}
                        \end{solution}
                \part[5] from the first-order perturbation formula for the 
                one-group approximation.
                        \begin{solution}
                            The first order pertubation formula in the one-group
                            approximation is
                            \begin{equation}
                                \Delta \rho = \frac{(D_0B_0^{2} + \Sigma_{a0})
                                \Delta \nu \Sigma_{f}}{(\nu\Sigma_{f0})^{2}} -
                                \frac{\Delta \Sigma_{a})}{\nu
                                \Sigma_{f0}}
                            \end{equation}
                            In this case, our pertubation is simply $\Delta
                            \Sigma_{a} = \delta \Sigma_{a}$. $\Delta \nu
                            \Sigma_{f} = 0$, so we get
                            \begin{equation}
                                \Delta \rho = -\frac{\delta \Sigma_{a}}{\nu
                                \Sigma_{f0}}
                            \end{equation}
                        \end{solution}
        \end{parts}

        % ---------------------------------------------
        \question[5] (Ott Review 5.4) What does ``exact'' mean in the context 
        of exact point kinetics equations?
        \begin{solution}
            ``Exact'' means we abandon the simplifying assumptions we
            made for one-group point kinetics (for example, we assume that
            flux is {\it not} separable in time). This will yield improved
            equations for our integral parameters ($\rho$, $\Lambda$, $\beta$)
            as well as power, $\zeta_{k}$, and $s(t)$.
        \end{solution}
        

        % ---------------------------------------------
        \question (Ott Problem 6.1) Inhour equation.
        \begin{parts}
                \part[10] Find the stable and prompt-period branches for 
                $^{235}U$  fuel with
                $\Lambda = 10^{-4}$s,  
                $\Lambda = 10^{-5}$s, and  
                $\Lambda = 4\times10^{-7}$s. Data is given in Table 2-III.
                \begin{solution}
                    The inhour equation is given by Equation 6.65 in Ott:
                    \begin{equation}
                        \rho(\alpha) = \alpha \Lambda + \sum_{k} \frac{\beta_{k}
                        \alpha}{\alpha + \lambda_{k}}
                    \end{equation}
                    Plugging in the given values for $\Lambda$ and $\lambda_{k}$
                    for $k=1,2,3,4,5,6$ to the  
                    The domain of the stable-period branch is bounded by
                    $-\lambda_{1}$ on the left, where $\lambda_{1}$ is the
                    smallest magnitude decay constant. This is because the
                    stable period is limited by the decay of the longest lived
                    isotopes. Likewise, the prompt-period branch is bounded by
                    $-\lambda_6$ on the right, where $\lambda_{6}$ is the
                    largest magnitude decay constant.
                \end{solution}

                \part[10] Find $\rho(\alpha)$ in the one-delay-group 
                approximation with $\lambda = \bar{\lambda}$.
                \begin{solution}
                    The one-delay group approximation is given by Equation 6.68
                    in Ott:
                    \begin{equation*}
                        \rho = \alpha\Lambda + \frac{\beta\alpha}{\alpha +
                        \lambda}
                    \end{equation*}
                    In terms of reactivity per $\beta$, this is
                    \begin{equation*}
                        \frac{\rho}{\beta} = \frac{\alpha\Lambda}{\beta} + \frac{\alpha}{\alpha +
                        \lambda}
                    \end{equation*}

                    We calculate $\overline{\lambda} = \frac{1}{6}\sum_{k=1}^{6}
                    \lambda_{k}$, so plugging in the values from Table 2-III in
                    Ott yields $\overline{\lambda} = 0.8298 s^{-1}$. For each
                    value of $\Lambda$, we have the following result:
                    \begin{itemize}
                        \item $\Lambda = 10^{-4}s$: $\frac{\rho}{\beta} =
                            \frac{\alpha 10^{-4}}{\beta} + \frac{\alpha}{\alpha + 0.8298}$
                        \item $\Lambda = 10^{-5}s$: $\frac{\rho}{\beta} =
                            \frac{\alpha 10^{-5}}{\beta} + \frac{\alpha}{\alpha + 0.8298}$
                        \item $\Lambda = 4 \times 10^{-7}s$: $\frac{\rho}{\beta} =
                            \frac{4\alpha 10^{-7}}{\beta} + \frac{\alpha}{\alpha + 0.8298}$
                    \end{itemize}
                \end{solution}

        \end{parts}

        % ---------------------------------------------
        \question[5] (Ott Review 6.13) Which conditions have to be fulfilled to 
        yield an asymptotic transient with a ``stable'' period?
        \begin{solution}
            Constant reactivitiy is required to establish an asymtotic transient
            with a stable period. According to Ott, this requires the power
            density ``to be small enough so that no noticeable reactivity
            feedback disturbing the constant reactivity develops''.
        \end{solution}

        
        % ---------------------------------------------
        \question[10] (Ott Review 6.18b) Estimate the stable period for an 
        asymptotic transient following a reactivity insertion of $\rho = 
        1.25\cent$.
        \begin{solution}
            A reactivity insertion of $1.25\cent = 0.0125\beta$ would lead to a slow
            transient, so we can assume that $\alpha \ll \lambda_{k}$. We can
            then apply Equation 6.76 from Ott:
            \begin{equation*}
                \alpha = \frac{\rho}{\beta} 0.08 s^{-1}
            \end{equation*}
            Plugging in the reactivity, we get
            \begin{align*}
                \alpha = 0.0125 * 0.08 = 0.001 s^{-1}
            \end{align*}
            Taking the reciprocal of $\alpha$, we get a stable period of 1000s.
        \end{solution}


        % ---------------------------------------------
        \question[10] (Ott Review 6.19) Give a list of six to seven 
        kinetics equations with decreasing sophistication of the description of 
        delayed neutrons.
        \begin{solution}
        \begin{itemize}
            \item 6 DNP group kinetics equation: $\dot{p} = \frac{\rho -
                \beta}{\Lambda}p + \frac{1}{\Lambda} \sum_{k=1}^{6} \lambda_{k}
                \zeta_{k}(t)$
            \item 2 DNP group kinetics equation: $\dot{p} = \frac{\rho -
                \beta}{\Lambda}p + \frac{1}{\Lambda} \sum_{k=1}^{2} \lambda_{k}
                \zeta_{k}(t)$ 
            \item 1 DNP group kinetics equation: $\dot{p} = \frac{\rho -
                    \beta}{\Lambda}p + \frac{1}{\Lambda} \overline{\lambda}
                \zeta(t)$ 
            \item Precursor accumulation (PA) approximation: $\dot{p} = \frac{\rho -
                \beta}{\Lambda}p + \frac{1}{\Lambda} [\beta_{0}p_{0} + \overline{\lambda}
                \beta I(t)]$
            \item Constant delayed source (CDS) approximation: $\dot{p} = \frac{\rho -
                \beta}{\Lambda}p + \frac{1}{\Lambda} \beta_{0}p_{0}$
            \item Prompt kinetics approximation: $\dot{p} = \frac{\rho -
                \beta}{\Lambda}p$
            \item Kinetics without delayed neutrons: $\dot{p} = \frac{\rho}{\Lambda}p$
        \end{itemize}
        \end{solution}

        \question (Ott Problem 3.2)
        Find $\overline{v}$ and $\overline{\frac{1}{v}}$ for a two-group
        representation of a thermal reactor spectrum, composed for simplicity of
        a Maxwellian and a 1/E spectrum:
        \begin{align*}
            \phi_1(E) &= \frac{a}{E} &\text{for 0.2eV $\leq E \leq 2$ MeV}\\ 
            \phi_2(E) &= \frac{bE}{(kT)^{2}}e^{-\frac{E}{kT}} &\text{for 0 $\leq
            E \leq \infty$}
        \end{align*}
        The Maxewllian decreases fast enough so that no finite upper limit needs
        to be considered in $\phi_2(E)$,
        \begin{parts}
                \part[3] Find $a$ and $b$ such that the two components of the
                normalized $\phi(E)$ provide equal contributions to the energy
                integral.
                \begin{solution}
                    Suppose $\phi(E) = \phi_1(E) + \phi_2(E)$ is normalized. We
                    then have that
                    \begin{align*}
                        1 &= \int_{0}^{\infy} \phi(E) dE = \int_{0.2}^{2e6}
                            \phi_1(E) dE + \int_{0}^{\infty} \phi_2(E) dE \\
                        &= \int_{0.2}^{2e6} \frac{a}{E} dE + \int_{0}^{\infty}
                          \frac{bE}{(kT)^{2}} e^{-\frac{E}{kT}} dE \\
                        &= a \ln(E)\Bigg|_{0.2}^{2e6} - \frac{be^{-\frac{E}{kT}}
                            (kT + E)}{kT} \Bigg|_{0}^{\infty}\\
                        &= a \ln{10^{7}} + b = a 7\ln(10) + b
                    \end{align*}
                    Now suppose that $\phi_1(E)$ and $\phi_2(E)$ provide equal
                    controbutions to the energy integral, that is, that
                    $\int_{0.2}^{2e6} \phi_1(E) dE + \int_{0}^{\infty} \phi_2(E)
                    dE$. We then have that $a 7 \ln{10} = b$. Substituting
                    this for $b$ in the previous expression and rearranging, we
                    get
                    \[
                        \boxed{
                        a = \frac{1}{14 \ln(10)}
                        }
                    \]
                    and then we get 
                    \[
                        \boxed{
                        b = 0.5
                        }
                    \]
                \end{solution}

                \part[3] Find the average velocities for both groups
                ($\overline{v}_{1}$ and $\overline{v}_{2}$).
                \begin{solution}
                    To calculate $\overline{v}$, the average velocity, we
                    compute
                    \begin{equation*}
                        \overline{v} =
                        \sqrt{\frac{2}{m_{n}}}\frac{\int_{0}^{\infty} \sqrt{E}
                        \phi(E) dE}{\int_{0}^{\infty} \phi(E) dE}
                    \end{equation*}
                    For group 1, this is
                    \begin{align*}
                        \overline{v}_1 &=
                            \sqrt{\frac{2}{m_{n}}}\frac{\int_{0.2}^{2e6}
                            \frac{a}{\sqrt{E}} dE}{\int_{0.2}^{2e6} \frac{a}{E}
                            dE}\\
                        &= \sqrt{\frac{2}{m_{n}}}
                            \frac{2a\sqrt{E}\Big|_{0.2}^{2e6}}{a
                            \ln(E)\Big|_{0.2}^{2e6}} \\
                    \end{align*}
                    After plugging in the values, I get an answer that is way
                    too big when using $m_{n} = 1.675e-27$ kg. Not sure what is
                    going on, but don't have time to spare to fix it. We'd
                    follow the same approach to get $\overline{v}_{2}$ but using
                    $\phi_{2}$.
                \end{solution}

                \part[2] Derive two-group definitions of $\overline{v}$ and
                $\overline{\frac{1}{v}}$.
                \begin{solution}
                    solution
                \end{solution}

                \part[2] Find the corresponding numerical values.
                \begin{solution}
                    solution
                \end{solution}
        \end{parts}


        
\end{questions}



%\bibliographystyle{plain}
%\bibliography{hw01}
\end{document}
