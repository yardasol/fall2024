% use the answers clause to get answers to print; otherwise leave it out.
\documentclass[11pt,addpoints,answers]{exam}
%\documentclass[11pt,addpoints]{exam}
\RequirePackage{amssymb, amsfonts, amsmath, latexsym, verbatim, xspace, 
setspace, wasysym, mathrsfs}
\usepackage{graphicx}

% By default LaTeX uses large margins.  This doesn't work well on exams; problems
% end up in the "middle" of the page, reducing the amount of space for students
% to work on them.
\usepackage[margin=1in]{geometry}
\usepackage{enumerate}
\usepackage[hidelinks]{hyperref}

% Here's where you edit the Class, Exam, Date, etc.
\newcommand{\class}{NPRE 560}
\newcommand{\term}{Fall 2024}
\newcommand{\assignment}{HW 5}
\newcommand{\duedate}{2024.12.09}
%\newcommand{\timelimit}{50 Minutes}
\newcommand{\StudentName}{Oleksandr Yardas} %Please include your name here

\newcommand{\nth}{n\ensuremath{^{\text{th}}} }
\newcommand{\ve}[1]{\ensuremath{\mathbf{#1}}}
\newcommand{\Macro}{\ensuremath{\Sigma}}
\newcommand{\vOmega}{\ensuremath{\hat{\Omega}}}

% For an exam, single spacing is most appropriate
\singlespacing
% \onehalfspacing
% \doublespacing

% For an exam, we generally want to turn off paragraph indentation
\parindent 0ex

%\unframedsolutions

\begin{document} 

% These commands set up the running header on the top of the exam pages
\pagestyle{head}
\firstpageheader{}{}{}
\runningheader{\class}{\assignment\ - Page \thepage\ of \numpages}{Due \duedate}
\runningheadrule

\class \hfill \StudentName \hfill \term \\
\assignment \hfill Due \duedate\\
\rule[1ex]{\textwidth}{.1pt}
%\hrulefill

%%%%%%%%%%%%%%%%%%%%%%%%%%%%%%%%%%%%%%%%%%%%%%%%%%%%%%%%%%%%%%%%%%%%%%%%%%%%%%%%%%%%%
%%%%%%%%%%%%%%%%%%%%%%%%%%%%%%%%%%%%%%%%%%%%%%%%%%%%%%%%%%%%%%%%%%%%%%%%%%%%%%%%%%%%%
\begin{itemize}
        \item Show your work. 
        \item This work must be submitted online as a \texttt{.pdf} through Canvas.
        \item Work completed with LaTeX or Jupyter earns 1 extra point. Submit 
                source file (e.g. \texttt{.tex} or \texttt{.ipynb}) along with 
                the \texttt{.pdf} file.
        \item If this work is completed with the aid of a numerical program 
                (such as Python, Wolfram Alpha, or MATLAB) all scripts and data 
                must be submitted in addition to the \texttt{.pdf}.
        \item If you work with anyone else, document what you worked on together.
\end{itemize}
\rule[1ex]{\textwidth}{.1pt}

% ---------------------------------------------
\begin{questions}

        % ---------------------------------------------
        \question[20] Give two reasons that make space-energy dependent 
        dynamics necessary. 
                \begin{solution}
                    \begin{enumerate}
                        \item From a physics perspective, point kinetics cannot
                            account for localized effects in a reactor core,
                            which leads to under-prediction of positive
                            reactivity insertions and over-prediction of
                            negative reactivity insertions. To model localized
                            effects accurately, we needs space-energy dynamics.
                            accurately, we need space-energy dependence.
                            %% This is the same as my answer to question 3?
                        \item From a safety perspective, as the assumptions of
                            point kinetics give inaccurate results on whole
                            cores, particular thermal reactor cores, they are
                            insufficient for safety studies. To model transients
                            accurately to determine reactor safety, we needs
                            space-energy dependence.
                    \end{enumerate}
                \end{solution}


        % ---------------------------------------------
        \question[10] Give an example of a scenario in which energy dependent 
        dynamics is necessary for reactor analysis.
                \begin{solution}
                    Consider a transient in a fast reactor. Fast reactors
                    generally have a wider neutron spectrum for fissioning
                    neutrons and are more sensitive to doppler broadening. To
                    account for the doppler feedback and neutron spectrum during
                    a transient in a fast reactor, energy dependence is needed.
                \end{solution}



        % ---------------------------------------------
        \question[10] Give an example of a scenario in which space dependent 
        dynamics is necessary for reactor analysis.
                \begin{solution}
                    Consider a transient in a thermal reactor. Thermal reactor
                    cores are very large, and thus have have highly localized
                    effects, particularly during a transient. Space dependence
                    is needed to capture these localized effects.
                \end{solution}




        % ---------------------------------------------
        \question Describe and discuss the advantages and disadvantages of the 
        following four approaches to space-energy dependent dynamics:
        \begin{parts}
                \part[10] the finite difference solution approach
                \begin{solution}
                    \begin{itemize}
                        \item Advantages: Simple mathematics; straightforward to
                            implement; accurate for sufficiently small timesteps
                        \item Disadvantages: Memory intensive; not fast.
                    \end{itemize}
                \end{solution}
                \part[10] the nodal approach
                \begin{solution}
                    \begin{itemize}
                        \item Advantages: ? 
                        \item Disadvantages: Expensive for complicated
                            geometries
                    \end{itemize}
                \end{solution}
                \part[10] the modal approach
                \begin{solution}
                    \begin{itemize}
                        \item Advantages: Modes only need to be calculated once
                            for a given problem, typically using a
                            lower-dimensional static problem
                        \item Disadvantages: Savings over finite differences
                            only achieved with a small number of modes
                    \end{itemize}
                \end{solution}
                \part[10] the quasistatic approach
                \begin{solution}
                    \begin{itemize}
                        \item Advantages: Relatively fast; Flux factorization
                            into shape and amplitude functions assumes shape
                            changes slowly with respect to time, so the shape
                            function is not recalculated at every timestep as
                            the amplitude function.
                        \item Disadvantages: Inaccuracies when simulating
                            initial fast reactivity changes 
                    \end{itemize}
                \end{solution}
        \end{parts}


        % ---------------------------------------------
        \question Discuss the relation of the quasistatic and the adiabatic 
        methods.
        \begin{parts}
                \part[10] Which approximation is common to both?
                \begin{solution}
                    Both methods assume the flux $\phi(\mathbf{r}, E, t)$ is
                    factored into a shape function, $\psi(\mathbf{r}, E, t)$,
                    and an amplitude function $p(t)$.
                \end{solution}
                \part[10] What are the key differences?
                \begin{solution}
                    The quasistatic method neglects the shape function time
                    derivative (in the improved quasistatic method, a finite
                    difference approximates the shape function time derivative).
                    In the adiabatic method, the $\frac{\alpha}{v}$ term is
                    neglected and the delayed neutron source is approximated as
                    a fraction of the prompt neutron source. This leads to the
                    adiabatic method eliminating kinetic features in the shape
                    function
                \end{solution}
        \end{parts}

\end{questions}



%\bibliographystyle{plain}
%\bibliography{hw01}
\end{document}
