% use the answers clause to get answers to print; otherwise leave it out.
\documentclass[11pt,addpoints,answers]{exam}
%\documentclass[11pt,addpoints]{exam}
\RequirePackage{amssymb, amsfonts, amsmath, latexsym, verbatim, xspace, 
setspace, wasysym, mathrsfs}
\usepackage{graphicx}

% By default LaTeX uses large margins.  This doesn't work well on exams; problems
% end up in the "middle" of the page, reducing the amount of space for students
% to work on them.
\usepackage[margin=1in]{geometry}
\usepackage{enumerate}
\usepackage[hidelinks]{hyperref}

% Here's where you edit the Class, Exam, Date, etc.
\newcommand{\class}{NPRE 560}
\newcommand{\term}{Fall 2024}
\newcommand{\assignment}{HW 5}
\newcommand{\duedate}{2024.12.09}
%\newcommand{\timelimit}{50 Minutes}
\newcommand{\StudentName}{Firstname Lastname} %Please include your name here

\newcommand{\nth}{n\ensuremath{^{\text{th}}} }
\newcommand{\ve}[1]{\ensuremath{\mathbf{#1}}}
\newcommand{\Macro}{\ensuremath{\Sigma}}
\newcommand{\vOmega}{\ensuremath{\hat{\Omega}}}

% For an exam, single spacing is most appropriate
\singlespacing
% \onehalfspacing
% \doublespacing

% For an exam, we generally want to turn off paragraph indentation
\parindent 0ex

%\unframedsolutions

\begin{document} 

% These commands set up the running header on the top of the exam pages
\pagestyle{head}
\firstpageheader{}{}{}
\runningheader{\class}{\assignment\ - Page \thepage\ of \numpages}{Due \duedate}
\runningheadrule

\class \hfill \StudentName \hfill \term \\
\assignment \hfill Due \duedate\\
\rule[1ex]{\textwidth}{.1pt}
%\hrulefill

%%%%%%%%%%%%%%%%%%%%%%%%%%%%%%%%%%%%%%%%%%%%%%%%%%%%%%%%%%%%%%%%%%%%%%%%%%%%%%%%%%%%%
%%%%%%%%%%%%%%%%%%%%%%%%%%%%%%%%%%%%%%%%%%%%%%%%%%%%%%%%%%%%%%%%%%%%%%%%%%%%%%%%%%%%%
\begin{itemize}
        \item Show your work. 
        \item This work must be submitted online as a \texttt{.pdf} through Canvas.
        \item Work completed with LaTeX or Jupyter earns 1 extra point. Submit 
                source file (e.g. \texttt{.tex} or \texttt{.ipynb}) along with 
                the \texttt{.pdf} file.
        \item If this work is completed with the aid of a numerical program 
                (such as Python, Wolfram Alpha, or MATLAB) all scripts and data 
                must be submitted in addition to the \texttt{.pdf}.
        \item If you work with anyone else, document what you worked on together.
\end{itemize}
\rule[1ex]{\textwidth}{.1pt}

% ---------------------------------------------
\begin{questions}

        % ---------------------------------------------
        \question[20] Give two reasons that make space-energy dependent 
        dynamics necessary. 
                \begin{solution}
                        Two reasons
                \end{solution}


        % ---------------------------------------------
        \question[10] Give an example of a scenario in which energy dependent 
        dynamics is necessary for reactor analysis.
                \begin{solution}
                        scenario
                \end{solution}



        % ---------------------------------------------
        \question[10] Give an example of a scenario in which space dependent 
        dynamics is necessary for reactor analysis.
                \begin{solution}
                        scenario
                \end{solution}




        % ---------------------------------------------
        \question Describe and discuss the advantages and disadvantages of the 
        following four approaches to space-energy dependent dynamics:
        \begin{parts}
                \part[10] the finite difference solution approach
                \begin{solution}
                        advantages and disadvantages.
                \end{solution}
                \part[10] the nodal approach
                \begin{solution}
                        advantages and disadvantages.
                \end{solution}
                \part[10] the modal approach
                \begin{solution}
                        advantages and disadvantages.
                \end{solution}
                \part[10] the quasistatic approach
                \begin{solution}
                        advantages and disadvantages.
                \end{solution}
        \end{parts}


        % ---------------------------------------------
        \question Discuss the relation of the quasistatic and the adiabatic 
        methods.
        \begin{parts}
                \part[10] Which approximation is common to both?
                \begin{solution}
                        solution
                \end{solution}
                \part[10] What are the key differences?
                \begin{solution}
                        solution
                \end{solution}
        \end{parts}

\end{questions}



%\bibliographystyle{plain}
%\bibliography{hw01}
\end{document}
